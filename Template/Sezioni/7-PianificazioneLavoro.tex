%----------------------------------------------------------------------------------------
%	PLANNING
%----------------------------------------------------------------------------------------

\section*{Pianificazione del lavoro}

\subsection*{Notazione utilizzata}

Di seguito viene presentata la pianificazione settimanale delle ore lavorative previste.
Ad ogni settimana sono assegnate le voci contenenti gli incrementi previsti in essa.
Ad ogni incremento è associato un requisito obbligatorio, desiderabile o facoltativo.
A questi requisiti vi è associato un codice identificativo per favorirne il tracciamento futuro, in che precede la voce descrittiva dell'incremento.
Ogni codice è composto da una lettera seguita da dei numeri interi, secondo il seguente modello:
\begin{center}
	\textbf{A-X.Y.Z}
\end{center}
ove, da sinistra verso destra:
\begin{itemize}

  \item \textbf{A} rappresenta la lettera che qualifica il requisito come obbligatorio, desiderabile o facoltativo, secondo la seguente notazione:
  \begin{itemize}
  	\item \textit{O} per i requisiti obbligatori, vincolanti in quanto obiettivo primario richiesto dal committente;
  	\item \textit{D} per i requisiti desiderabili, non vincolanti o strettamente necessari,
  		  ma dal riconoscibile valore aggiunto;
  	\item \textit{F} per i requisiti facoltativi, rappresentanti valore aggiunto non strettamente
  		  competitivo.
  \end{itemize}

  \item \textbf{X} rappresenta la settimana in cui viene inizialmente pianificato l'incremento (identificata da un numero incrementale e intero, partendo da 1).
  Questo consente allo studente, al tutor interno e al tutor interno una rapida quantificazione dell'avanzamento corrente dello stage rispetto a quanto inizialmente pianificato.

  \item \textbf{Y} rappresenta la posizione sequenziale prevista dell'incremento all'interno della settimana (incrementale e intero, partendo da 1). Esso è strettamente associato alla lettera, ovvero l'incremento di questo numero dipende esclusivamente dal

  \item \textbf{Z} rappresenta l'eventuale suddivisione in sottoincrementi (incrementale e intero, partendo da 1; può essere assente).
\end{itemize}
\newpage
 % o ed è composta da tre numeri che possono assumere valori interi, con incremento di una singola unità alla volta. Questi tre valori sono separati da punti nel seguente modo:

\subsection*{Pianificazione settimanale}
\prospettoSettimanale

\newpage

\subsection*{Ripartizione ore}

La pianificazione, in termini di quantità di ore di lavoro, sarà così distribuita:

\begin{center}
    
% Tabella da personalizzare in base alle ore delle attività

\begin{tabularx}{\textwidth}{|c|X|}
	\hline
	\textbf{Durata in ore} & \textbf{Descrizione dell'attività} \\\hline

		\textbf{240} & \textbf{Formazione sulle tecnologie} \\
		\hdashline
		\multirow{3}{0cm}\\
		\textit{80} &
		\textit{Formazione e documentazione riguardo i concetti chiave dell'area EAI} \\
		\textit{80} &
		\textit{Formazione riguardo i principali software Middleware} \\
		\textit{80} &
		\textit{Studio di Apache Kafka e Confluent Platform} \\
    \hline

    \textbf{20} & \textbf{Progettazione del confronto, della configurazione e della documentazione} \\ \hdashline
    \multirow{4}{0cm}\\
    \textit{4} &
    \textit{Analisi del problema e del dominio applicativo} \\
		\textit{10} &
		\textit{Progettazione dell'analisi comparativa} \\
    \textit{3} &
    \textit{Progettazione della configurazione di Apache Kafka} \\
    \textit{3} &
    \textit{Stesura documentazione relativa ad analisi e progettazione} \\
    \hline

		\textbf{20} & \textbf{Analisi comparativa e relativa documentazione}  \\ \hdashline
    \multirow{4}{0cm}\\
		\textit{12} &
		\textit{Analisi comparativa fra i principali ESB e Apache Kafka} \\
    \textit{8} &
    \textit{Documentazione relativa} \\
    \hline

    \textbf{20} & \textbf{Valutazioni finali e conclusione}  \\ \hdashline
    \multirow{4}{0cm}\\
		\textit{5} &
		\textit{Revisione e validazione della documentazione prodotta} \\
    \textit{8} &
    \textit{Valutazioni finali} \\
    \textit{5} &
    \textit{Conclusione} \\
    \textit{2} &
    \textit{Preparazione e presentazione del prodotto finale con gli stakeholders} \\
    \hline

		\textbf{Totale delle ore} & \textbf{300} \\\hline


\end{tabularx}

\end{center}

\newpage
