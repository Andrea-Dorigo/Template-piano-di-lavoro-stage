%----------------------------------------------------------------------------------------
%   USEFUL COMMANDS
%----------------------------------------------------------------------------------------

\newcommand{\dipartimento}{Dipartimento di Matematica ``Tullio Levi-Civita''}

%----------------------------------------------------------------------------------------
% 	USER DATA
%----------------------------------------------------------------------------------------

% Data di approvazione del piano da parte del tutor interno; nel formato GG Mese AAAA
% compilare inserendo al posto di GG 2 cifre per il giorno, e al posto di
% AAAA 4 cifre per l'anno
\newcommand{\dataApprovazione}{Data}

% Dati dello Studente
\newcommand{\nomeStudente}{Andrea}
\newcommand{\cognomeStudente}{Dorigo}
\newcommand{\matricolaStudente}{1170610}
\newcommand{\emailStudente}{andrea.dorigo.3@studenti.unipd.it}
\newcommand{\telStudente}{+ 39 331 40 98 277}

% Dati del Tutor Aziendale
\newcommand{\nomeTutorAziendale}{Francesco Giovanni}
\newcommand{\cognomeTutorAziendale}{Sanges}
\newcommand{\emailTutorAziendale}{f.sanges@synclab.it}
\newcommand{\telTutorAziendale}{+ 39 000 00 00 000}
\newcommand{\ruoloTutorAziendale}{Tutor esterno area EAI}

% Dati dell'Azienda
\newcommand{\ragioneSocAzienda}{Sync Lab Srl}
\newcommand{\indirizzoAzienda}{Via G. Porzio, CDN is. B8, 80143 Napoli}
\newcommand{\sitoAzienda}{https://www.synclab.it}
\newcommand{\emailAzienda}{info@synclab.it}
\newcommand{\partitaIVAAzienda}{P.IVA 07952560634}

% Dati del Tutor Interno (Docente)
\newcommand{\titoloTutorInterno}{Prof.}
\newcommand{\nomeTutorInterno}{Tullio}
\newcommand{\cognomeTutorInterno}{Vardanega}

\newcommand{\prospettoSettimanale}{
     % Personalizzare indicando in lista, i vari task settimana per settimana
     % sostituire a XX il totale ore della settimana
    \begin{itemize}
        \item \textbf{Prima Settimana (40 ore)}
        \begin{itemize}
            \item Incontro con le persone coinvolte nel progetto per discutere i requisiti e le richieste relative al sistema da sviluppare;
            \item Verifica credenziali e strumenti di lavoro assegnati;
            \item Presa visione dell’infrastruttura esistente;
            \item Ripasso approfondito riguardo i seguenti argomenti;
            \begin{itemize}
              \item Ingegneria del Software
              \item Sistemi di Versionamento
              \item Networking
              \item Architetture Software
            \end{itemize}
            \item Documentazione e formazione riguardo i seguenti argomenti:
            \begin{itemize}
              \item Enterprise Application Integration Patterns
              \item Apache Kafka
            \end{itemize}

        \end{itemize}
        \item \textbf{Seconda Settimana - Sottotitolo (40 ore)}
        \begin{itemize}
            \item ;
        \end{itemize}
        \item \textbf{Terza Settimana - Sottotitolo (40 ore)}
        \begin{itemize}
            \item ;
        \end{itemize}
        \item \textbf{Quarta Settimana - Sottotitolo (40 ore)}
        \begin{itemize}
            \item ;
        \end{itemize}
        \item \textbf{Quinta Settimana - Sottotitolo (40 ore)}
        \begin{itemize}
            \item ;
        \end{itemize}
        \item \textbf{Sesta Settimana - Sottotitolo (40 ore)}
        \begin{itemize}
            \item ;
        \end{itemize}
        \item \textbf{Settima Settimana - Sottotitolo (40 ore)}
        \begin{itemize}
            \item ;
        \end{itemize}
        \item \textbf{Ottava Settimana - Conclusione (20 ore)}
        \begin{itemize}
            \item ;
        \end{itemize}
    \end{itemize}
}

% Indicare il totale complessivo (deve essere compreso tra le 300 e le 320 ore)
\newcommand{\totaleOre}{}

\newcommand{\obiettiviObbligatori}{
	 \item \underline{\textit{O01}}: primo obiettivo;
	 \item \underline{\textit{O02}}: secondo obiettivo;
	 \item \underline{\textit{O03}}: terzo obiettivo;

}

\newcommand{\obiettiviDesiderabili}{
	 \item \underline{\textit{D01}}: primo obiettivo;
	 \item \underline{\textit{D02}}: secondo obiettivo;
}

\newcommand{\obiettiviFacoltativi}{
	 \item \underline{\textit{F01}}: primo obiettivo;
	 \item \underline{\textit{F02}}: secondo obiettivo;
	 \item \underline{\textit{F03}}: terzo obiettivo;
}
