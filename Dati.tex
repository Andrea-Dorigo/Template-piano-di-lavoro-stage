%----------------------------------------------------------------------------------------
%   USEFUL COMMANDS
%----------------------------------------------------------------------------------------

\newcommand{\dipartimento}{Dipartimento di Matematica ``Tullio Levi-Civita''}

%----------------------------------------------------------------------------------------
% 	USER DATA
%----------------------------------------------------------------------------------------

% Data di approvazione del piano da parte del tutor interno; nel formato GG Mese AAAA
% compilare inserendo al posto di GG 2 cifre per il giorno, e al posto di
% AAAA 4 cifre per l'anno
\newcommand{\dataApprovazione}{Data}

% Dati dello Studente
\newcommand{\nomeStudente}{Andrea}
\newcommand{\cognomeStudente}{Dorigo}
\newcommand{\matricolaStudente}{1170610}
\newcommand{\emailStudente}{andrea.dorigo.3@studenti.unipd.it}
\newcommand{\telStudente}{+ 39 331 40 47 431}

% Dati del Tutor Aziendale
\newcommand{\nomeTutorAziendale}{Francesco Giovanni}
\newcommand{\cognomeTutorAziendale}{Sanges}
\newcommand{\emailTutorAziendale}{f.sanges@synclab.it}
\newcommand{\telTutorAziendale}{+ 39 338 78 55 236}
\newcommand{\ruoloTutorAziendale}{Tutor esterno area EAI}

% Dati dell'Azienda
\newcommand{\ragioneSocAzienda}{Sync Lab Srl}
\newcommand{\indirizzoAzienda}{Via G. Porzio, CDN is. B8, 80143 Napoli}
\newcommand{\sitoAzienda}{https://www.synclab.it}
\newcommand{\emailAzienda}{info@synclab.it}
\newcommand{\partitaIVAAzienda}{P.IVA 07952560634}

% Dati del Tutor Interno (Docente)
\newcommand{\titoloTutorInterno}{Prof.}
\newcommand{\nomeTutorInterno}{Tullio}
\newcommand{\cognomeTutorInterno}{Vardanega}

\newcommand{\prospettoSettimanale}{
     % Personalizzare indicando in lista, i vari task settimana per settimana
     % sostituire a XX il totale ore della settimana

    \begin{itemize}
        \item \textbf{Prima Settimana - Introduzione e studio individuale (40 ore)}
        \begin{itemize}
            \item \textbf{O-1.1} Incontro con le persone coinvolte nel progetto per discutere i requisiti e le richieste relative al sistema da sviluppare;
            \item \textbf{O-1.2} Verifica credenziali e strumenti di lavoro assegnati;
            \item \textbf{O-1.3} Presa visione dell’infrastruttura esistente;
            \item \textbf{D-1.1} Ripasso approfondito riguardo i seguenti argomenti:
              \begin{itemize}
                \item \textbf{D-1.1.1} Ingegneria del Software;
                \item \textbf{D-1.1.2} Sistemi di Versionamento;
                \item \textbf{D-1.1.3} Architetture Software;
                \item \textbf{D-1.1.4} Cenni di Networking.
              \end{itemize}
        \end{itemize}


        \item \textbf{Seconda Settimana - Studio individuale sui principali Enterprise Integration  Messaging Pattern (40 ore)}
        \begin{itemize}
            \item \textbf{O-2.1} Nozioni fondamentali sul Enterprise Application Integration
            \item \textbf{O-2.2} Nozioni fondamentali sulla Service-Oriented Architecture
            \item \textbf{O-2.3} Approfondimenti riguardo le architetture a Messaggio, fra cui:
              \begin{itemize}
                \item \textbf{O-2.3.1} Integration Styles;
                \item \textbf{O-2.3.2} Channel Patterns;
                \item \textbf{O-2.3.3} Message Construction Patterns;
                \item \textbf{O-2.3.3} Routing Patterns;
                \item \textbf{O-2.3.4} Transformation Patterns;
                \item \textbf{O-2.3.5} System Management Patterns.
              \end{itemize}
        \end{itemize}


        \item \textbf{Terza Settimana - Approfondimento riguardo i software Middleware (40 ore)}
        \begin{itemize}
            \item \textbf{O-3.1} Cenni e casi d'uso riguardo i software Middleware più comuni:
              \begin{itemize}
                \item \textbf{O-3.1.1} Suite TIBCO;
                \item \textbf{O-3.1.2} WSO2;
              \end{itemize}
        \end{itemize}


        \item \textbf{Quarta Settimana - Approfondimento riguardo i software Middleware (40 ore)}
        \begin{itemize}
            \item \textbf{O-4.1} Cenni e casi d'uso riguardo i software Middleware più comuni:
              \begin{itemize}
                \item \textbf{O-4.1.1} Mule ESB;
              \end{itemize}
            \item \textbf{D-4.1} Cenni e casi d'uso riguardo i software Middleware più comuni:
              \begin{itemize}
                \item \textbf{D-4.1.1} Apache Camel;
              \end{itemize}
        \end{itemize}

        \newpage

        \item \textbf{Quinta Settimana - Apache Kafka (40 ore)}
        \begin{itemize}
          \item \textbf{O-5.1} Apache Kafka:
            \begin{itemize}
              \item \textbf{O-5.1.1} Introduzione a Kafka;
              \item \textbf{O-5.1.2} Concetti fondamentali di Kafka;
              \item \textbf{O-5.1.3} Avvio e Command Line Interface;
              \item \textbf{O-5.1.4} Programmazione in Kafka con Java;
            \end{itemize}
            \item \textbf{D-5.1} Esempi e Applicazioni di Apache Kafka:
        \end{itemize}


        \item \textbf{Sesta Settimana - Confluent (40 ore)}
        \begin{itemize}
          \item \textbf{O-6.1} Confluent Platform:
            \begin{itemize}
              \item \textbf{O-6.1.1} Service Registry;
              \item \textbf{O-6.1.2} REST proxy;
              \item \textbf{O-6.1.3} kSQL;
              \item \textbf{O-6.1.4} Confluent connectors;
              \item \textbf{O-6.1.5} Control Center.
            \end{itemize}
        \end{itemize}


        \item \textbf{Settima Settimana - Analisi comparativa (40 ore)}
        \begin{itemize}
            \item \textbf{O-7.1} Analisi comparativa fra gli ESB e le potenzialità offerte da Kafka.
        \end{itemize}


        \item \textbf{Ottava Settimana - Conclusioni (20 ore)}
        \begin{itemize}
            \item \textbf{O-8.1} Valutazioni finali e conclusione;
            \item \textbf{O-8.2} Controllo e validazione della documentazione prodotta.
        \end{itemize}

    \end{itemize}
}

% Indicare il totale complessivo (deve essere compreso tra le 300 e le 320 ore)
\newcommand{\totaleOre}{}

\newcommand{\obiettiviObbligatori}{
	 \item \underline{\textit{O01}}: primo obiettivo;
	 \item \underline{\textit{O02}}: secondo obiettivo;
	 \item \underline{\textit{O03}}: terzo obiettivo;

}

\newcommand{\obiettiviDesiderabili}{
	 \item \underline{\textit{D01}}: primo obiettivo;
	 \item \underline{\textit{D02}}: secondo obiettivo;
}

\newcommand{\obiettiviFacoltativi}{
	 \item \underline{\textit{F01}}: primo obiettivo;
	 \item \underline{\textit{F02}}: secondo obiettivo;
	 \item \underline{\textit{F03}}: terzo obiettivo;
}
